%% FEUP THESIS STYLE for LaTeX2e
%% how to use feupteses (portuguese version)
%%
%% FEUP, JCL & JCF, 31 Jul 2012
%%
%% PLEASE send improvements to jlopes at fe.up.pt and to jcf at fe.up.pt
%%

%%========================================
%% Commands: pdflatex tese
%%           bibtex tese
%%           makeindex tese (only if creating an index) 
%%           pdflatex tese
%% Alternative:
%%          latexmk -pdf tese.tex
%%========================================

\documentclass[11pt,a4paper,twoside,openright]{report}

%% For iso-8859-1 (latin1), comment next line and uncomment the second line
\usepackage[utf8]{inputenc}
%\usepackage[latin1]{inputenc}

%% Portuguese version

%% MIEIC options
\usepackage[portugues,mieic]{feupteses}
%\usepackage[portugues,mieic,juri]{feupteses}
%\usepackage[portugues,mieic,final]{feupteses}
%\usepackage[portugues,mieic,final,onpaper]{feupteses}

%% Options: 
%% - portugues: titles, etc in portuguese
%% - onpaper: links are not shown (for paper versions)
%% - backrefs: include back references from bibliography to citation place

%% Uncomment the next lines if side by side graphics used
%\usepackage[lofdepth,lotdepth]{subfig}
%\usepackage{graphicx}
%\usepackage{float}

%% Include color package
\usepackage{color}
\definecolor{cloudwhite}{cmyk}{0,0,0,0.025}

%% Include source-code listings package
\usepackage{listings}
\lstset{ %
 language=C,                        % choose the language of the code
 basicstyle=\footnotesize\ttfamily,
 keywordstyle=\bfseries,
 numbers=left,                      % where to put the line-numbers
 numberstyle=\scriptsize\texttt,    % the size of the fonts that are used for the line-numbers
 stepnumber=1,                      % the step between two line-numbers. If it's 1 each line will be numbered
 numbersep=8pt,                     % how far the line-numbers are from the code
 frame=tb,
 float=htb,
 aboveskip=8mm,
 belowskip=4mm,
 backgroundcolor=\color{cloudwhite},
 showspaces=false,                  % show spaces adding particular underscores
 showstringspaces=false,            % underline spaces within strings
 showtabs=false,                    % show tabs within strings adding particular underscores
 tabsize=2,	                    % sets default tabsize to 2 spaces
 captionpos=b,                      % sets the caption-position to bottom
 breaklines=true,                   % sets automatic line breaking
 breakatwhitespace=false,           % sets if automatic breaks should only happen at whitespace
 escapeinside={\%*}{*)},            % if you want to add a comment within your code
 morekeywords={*,var,template,new}  % if you want to add more keywords to the set
}

%% Uncomment next line to set the depth of sectional units listed in the toc
%\setcounter{tocdepth}{3}

%% Uncomment to create an index (at the end of the document)
%\makeindex

%% Path to the figures directory
%% TIP: use folder ``figures'' to keep all your figures
\graphicspath{{figures/}}

%%----------------------------------------
%% TIP: if you want to define more macros, use an external file to keep them
%some macro definitions

% format
\newcommand{\class}[1]{{\normalfont\slshape #1\/}}

% entities
\newcommand{\Feup}{Faculdade de Engenharia da Universidade do Porto}

\newcommand{\svg}{\class{SVG}}
\newcommand{\scada}{\class{SCADA}}
\newcommand{\scadadms}{\class{SCADA/DMS}}

%%----------------------------------------

%%========================================
%% Start of document
%%========================================
\begin{document}

%%----------------------------------------
%% Information about the work
%%----------------------------------------
\title{Título da Dissertação}
\author{Nome do Autor}

%% Uncomment next line for date of submission
%\thesisdate{31 de Julho de 2008}

%% Uncomment next line for copyright text if used
%\copyrightnotice{Nome do Autor, 2008}

\supervisor{Orientador}{Nome do Orientador}

%% Uncomment next line if necessary
%\supervisor{Co-orientador}{Nome de Outro Orientador}

%% Uncomment committee stuff in the final version
%\committeetext{Aprovado em provas públicas pelo Júri:}
%\committeemember{Presidente}{Nome do presidente do júri}
%\committeemember{Arguente}{Nome do arguente do júri}
%\committeemember{Vogal}{Nome do vogal do júri}
%\signature

%% Specify cover logo (in folder ``figures'')
\logo{uporto-feup.pdf}

%% Uncomment next line for additional text below the author's name (front page)
\additionalfronttext{Preparação da Dissertação}

%%----------------------------------------
%% Preliminary materials
%%----------------------------------------

% remove unnecessary \include{} commands
\begin{Prolog}
  \chapter*{Resumo}

O Resumo fornece ao leitor um sumário do conteúdo da dissertação.
Deverá ser breve mas conter detalhe suficiente e, uma vez que é a porta
de entrada para a dissertação, deverá dar ao leitor uma boa impressão
inicial.

Este texto inicial da dissertação é escrito no fim e resume numa
página, sem referências externas, o tema e o contexto do trabalho, a
motivação e os objectivos, as metodologias e técnicas empregues, os
principais resultados alcançados e as conclusões.

Este documento ilustra o formato a usar em dissertações na \Feup.
São dados exemplos de margens, cabeçalhos, títulos, paginação, estilos
de índices, etc. 
São ainda dados exemplos de formatação de citações, figuras e tabelas,
equações, referências cruzadas, lista de referências e índices.
%Este documento não pretende exemplificar conteúdos a usar. 
É usado texto descartável, \emph{Loren Ipsum}, para preencher a
dissertação por forma a ilustrar os formatos.

Seguem-se umas notas breves mas muito importantes sobre a versão 
provisória e a versão final do documento. 
A versão provisória, depois de verificada pelo orientador e de 
corrigida em contexto pelo autor, deve ser publicada na página 
pessoal de cada estudante/dissertação, juntamente com os dois 
resumos, em português e em inglês; deve manter a marca da água, 
assim como a numeração de linhas conforme aqui se demonstra.

A versão definitiva, a produzir somente após a defesa, em versão 
impressa (dois exemplares com capas próprias FEUP) e em versão 
eletrónica (6 CDs com "rodela" própria FEUP), deve ser limpa da marca de 
água e da numeração de linhas e deve conter a identificação, na primeira 
página, dos elementos do júri respetivo. 
Deve ainda, se for o caso, ser corrigida de acordo com as instruções 
recebidas dos elementos júri.

Lorem ipsum dolor sit amet, consectetuer adipiscing elit. Sed vehicula
lorem commodo dui. Fusce mollis feugiat elit. Cum sociis natoque
penatibus et magnis dis parturient montes, nascetur ridiculus
mus. Donec eu quam. Aenean consectetuer odio quis nisi. Fusce molestie
metus sed neque. Praesent nulla. Donec quis urna. Pellentesque
hendrerit vulputate nunc. Donec id eros et leo ullamcorper
placerat. Curabitur aliquam tellus et diam. 

Ut tortor. Morbi eget elit. Maecenas nec risus. Sed ultricies. Sed
scelerisque libero faucibus sem. Nullam molestie leo quis
tellus. Donec ipsum. Nulla lobortis purus pharetra turpis. Nulla
laoreet, arcu nec hendrerit vulputate, tortor elit eleifend turpis, et
aliquam leo metus in dolor. Praesent sed nulla. Mauris ac augue. Cras
ac orci. Etiam sed urna eget nulla sodales venenatis. Donec faucibus
ante eget dui. Nam magna. Suspendisse sollicitudin est et mi. 

Phasellus ullamcorper justo id risus. Nunc in leo. Mauris auctor
lectus vitae est lacinia egestas. Nulla faucibus erat sit amet lectus
varius semper. Praesent ultrices vehicula orci. Nam at metus. Aenean
eget lorem nec purus feugiat molestie. Phasellus fringilla nulla ac
risus. Aliquam elementum aliquam velit. Aenean nunc odio, lobortis id,
dictum et, rutrum ac, ipsum. 

Ut tortor. Morbi eget elit. Maecenas nec risus. Sed ultricies. Sed
scelerisque libero faucibus sem. Nullam molestie leo quis
tellus. Donec ipsum. 

\chapter*{Abstract}

Here goes the abstract written in English.

Lorem ipsum dolor sit amet, consectetuer adipiscing elit. Sed vehicula
lorem commodo dui. Fusce mollis feugiat elit. Cum sociis natoque
penatibus et magnis dis parturient montes, nascetur ridiculus
mus. Donec eu quam. Aenean consectetuer odio quis nisi. Fusce molestie
metus sed neque. Praesent nulla. Donec quis urna. Pellentesque
hendrerit vulputate nunc. Donec id eros et leo ullamcorper
placerat. Curabitur aliquam tellus et diam. 

Ut tortor. Morbi eget elit. Maecenas nec risus. Sed ultricies. Sed
scelerisque libero faucibus sem. Nullam molestie leo quis
tellus. Donec ipsum. Nulla lobortis purus pharetra turpis. Nulla
laoreet, arcu nec hendrerit vulputate, tortor elit eleifend turpis, et
aliquam leo metus in dolor. Praesent sed nulla. Mauris ac augue. Cras
ac orci. Etiam sed urna eget nulla sodales venenatis. Donec faucibus
ante eget dui. Nam magna. Suspendisse sollicitudin est et mi. 

Fusce sed ipsum vel velit imperdiet dictum. Sed nisi purus, dapibus
ut, iaculis ac, placerat id, purus. Integer aliquet elementum
libero. Phasellus facilisis leo eget elit. Nullam nisi magna, ornare
at, aliquet et, porta id, odio. Sed volutpat tellus consectetuer
ligula. Phasellus turpis augue, malesuada et, placerat fringilla,
ornare nec, eros. Class aptent taciti sociosqu ad litora torquent per
conubia nostra, per inceptos himenaeos. Vivamus ornare quam nec sem
mattis vulputate. Nullam porta, diam nec porta mollis, orci leo
condimentum sapien, quis venenatis mi dolor a metus. Nullam
mollis. Aenean metus massa, pellentesque sit amet, sagittis eget,
tincidunt in, arcu. Vestibulum porta laoreet tortor. Nullam mollis
elit nec justo. In nulla ligula, pellentesque sit amet, consequat sed,
faucibus id, velit. Fusce purus. Quisque sagittis urna at quam. Ut eu
lacus. Maecenas tortor nibh, ultricies nec, vestibulum varius, egestas
id, sapien. 

Phasellus ullamcorper justo id risus. Nunc in leo. Mauris auctor
lectus vitae est lacinia egestas. Nulla faucibus erat sit amet lectus
varius semper. Praesent ultrices vehicula orci. Nam at metus. Aenean
eget lorem nec purus feugiat molestie. Phasellus fringilla nulla ac
risus. Aliquam elementum aliquam velit. Aenean nunc odio, lobortis id,
dictum et, rutrum ac, ipsum. 

Ut tortor. Morbi eget elit. Maecenas nec risus. Sed ultricies. Sed
scelerisque libero faucibus sem. Nullam molestie leo quis
tellus. Donec ipsum. Nulla lobortis purus pharetra turpis. Nulla
laoreet, arcu nec hendrerit vulputate, tortor elit eleifend turpis, et
aliquam leo metus in dolor. Praesent sed nulla. Mauris ac augue. Cras
ac orci. Etiam sed urna eget nulla sodales venenatis. Donec faucibus
ante eget dui. Nam magna. Suspendisse sollicitudin est et mi. 

Phasellus ullamcorper justo id risus. Nunc in leo. Mauris auctor
lectus vitae est lacinia egestas. Nulla faucibus erat sit amet lectus
varius semper. Praesent ultrices vehicula orci. Nam at metus. Aenean
eget lorem nec purus feugiat molestie. Phasellus fringilla nulla ac
risus. Aliquam elementum aliquam velit. Aenean nunc odio, lobortis id,
dictum et, rutrum ac, ipsum. 

Ut tortor. Morbi eget elit. Maecenas nec risus. Sed ultricies. Sed
scelerisque libero faucibus sem. Nullam molestie leo quis
tellus. Donec ipsum. 
 % the abstract
  \chapter*{Agradecimentos}
%\addcontentsline{toc}{chapter}{Agradecimentos}

Aliquam id dui. Nulla facilisi. Nullam ligula nunc, viverra a, iaculis
at, faucibus quis, sapien. Cum sociis natoque penatibus et magnis dis
parturient montes, nascetur ridiculus mus. Curabitur magna ligula,
ornare luctus, aliquam non, aliquet at, tortor. Donec iaculis nulla
sed eros. Sed felis. Nam lobortis libero. Pellentesque
odio. Suspendisse potenti. Morbi imperdiet rhoncus magna. Morbi
vestibulum interdum turpis. Pellentesque varius. Morbi nulla urna,
euismod in, molestie ac, placerat in, orci. 

Ut convallis. Suspendisse luctus pharetra sem. Sed sit amet mi in diam
luctus suscipit. Nulla facilisi. Integer commodo, turpis et semper
auctor, nisl ligula vestibulum erat, sed tempor lacus nibh at
turpis. Quisque vestibulum pulvinar justo. Class aptent taciti
sociosqu ad litora torquent per conubia nostra, per inceptos
himenaeos. Nam sed tellus vel tortor hendrerit pulvinar. Phasellus
eleifend, augue at mattis tincidunt, lorem lorem sodales arcu, id
volutpat risus est id neque. Phasellus egestas ante. Nam porttitor
justo sit amet urna. Suspendisse ligula nunc, mollis ac, elementum
non, venenatis ut, mauris. Mauris augue risus, tempus scelerisque,
rutrum quis, hendrerit at, nunc. Nulla posuere porta orci. Nulla dui. 

Fusce gravida placerat sem. Aenean ipsum diam, pharetra vitae, ornare
et, semper sit amet, nibh. Nam id tellus. Etiam ultrices. Praesent
gravida. Aliquam nec sapien. Morbi sagittis vulputate dolor. Donec
sapien lorem, laoreet egestas, pellentesque euismod, porta at,
sapien. Integer vitae lacus id dui convallis blandit. Mauris non
sem. Integer in velit eget lorem scelerisque vehicula. Etiam tincidunt
turpis ac nunc. Pellentesque a justo. Mauris faucibus quam id
eros. Cras pharetra. Fusce rutrum vulputate lorem. Cras pretium magna
in nisl. Integer ornare dui non pede. 

\vspace{10mm}
\flushleft{O Nome do Autor}
  % the acknowledgments
  \include{quote}    % initial quotation if desired
  \cleardoublepage
  \pdfbookmark[0]{Conteúdo}{contents}
  \tableofcontents
  \cleardoublepage
  \pdfbookmark[0]{Lista de Figuras}{figures}
  \listoffigures
  \cleardoublepage
  \pdfbookmark[0]{Lista de Tabelas}{tables}
  \listoftables
  \chapter*{Abreviaturas e Símbolos}
%\addcontentsline{toc}{chapter}{Abbreviations}
\chaptermark{ABREVIATURAS E SÍMBOLOS}

\begin{flushleft}
\begin{tabular}{l p{0.8\linewidth}}
ADT      & Abstract Data Type\\
ANDF     & Architecture-Neutral Distribution Format\\
API      & Application Programming Interface\\
CAD      & Computer-Aided Design\\
CASE     & Computer-Aided Software Engineering\\
CORBA    & Common Object Request Broker Architecture\\
UNCOL    & UNiversal COmpiler-oriented Language\\
Loren    & Lorem ipsum dolor sit amet, consectetuer adipiscing
elit. Sed vehicula lorem commodo dui\\
WWW      & \emph{World Wide Web}
\end{tabular}
\end{flushleft}

  % the list of abbreviations used
\end{Prolog}

%%----------------------------------------
%% Body
%%----------------------------------------

\StartBody

%% TIP: use a separate file for each chapter
\include{chapter1} 
\include{chapter2}
\include{chapter3}
\include{chapter4}
\chapter{Conclusões e Trabalho Futuro} \label{chap:concl}

\section*{}

Deve ser apresentado um resumo do trabalho realizado e apreciada a
satisfação dos objetivos do trabalho, uma lista de contribuições
principais do trabalho e as direções para trabalho futuro.

A escrita deste capítulo deve ser orientada para a total compreensão
do trabalho, tendo em atenção que, depois de ler o Resumo e a
Introdução, a maioria dos leitores passará à leitura deste capítulo de
conclusões e recomendações para trabalho futuro.

\section{Satisfação dos Objetivos}

Lorem ipsum dolor sit amet, consectetuer adipiscing elit. Etiam non
felis sed odio rutrum ultrices. Donec tempor dolor. Vivamus justo
neque, tempus id, ullamcorper in, pharetra non, tellus. Praesent eu
orci eu dolor congue gravida. Sed eu est. Donec pulvinar, lectus et
eleifend volutpat, diam sapien sollicitudin arcu, a sagittis libero
neque et dolor. Nam ligula. Cras tincidunt lectus quis nunc. Cras
tincidunt congue turpis. Nulla pede velit, sagittis a, faucibus vitae,
porttitor nec, ante. Nulla ut arcu. Cras eu augue at ipsum feugiat
hendrerit. Proin sed justo eu sapien eleifend elementum. Pellentesque
habitant morbi tristique senectus et netus et malesuada fames ac
turpis egestas. Vivamus quam lacus, pharetra vel, aliquam vel,
volutpat sed, nisl. 

Nullam erat est, vehicula id, tempor non, scelerisque at,
tellus. Pellentesque tincidunt, ante vehicula bibendum adipiscing,
lorem augue tempor felis, in dictum massa justo sed metus. Suspendisse
placerat, mi eget molestie sodales, tortor ante interdum dui, ac
sagittis est pede et lacus. Duis sapien. Nam ornare turpis et
magna. Etiam adipiscing adipiscing ipsum. Fusce sodales nisl a
arcu. Cras massa leo, vehicula facilisis, commodo a, molestie
faucibus, metus. Suspendisse potenti. Duis sagittis. Donec porta. Sed
urna. Maecenas eros. Vivamus erat ligula, pharetra sit amet, bibendum
et, fermentum sed, dolor. Nullam eleifend condimentum nibh. Integer
leo nibh, consequat eget, mollis et, sagittis ac, felis. Duis viverra
pede in pede. Phasellus molestie placerat leo. Praesent at tellus a
augue congue molestie. Proin sed justo eu sapien eleifend
elementum. Pellentesque habitant morbi tristique senectus et netus et
malesuada fames ac turpis egestas. 

\section{Trabalho Futuro}

Lorem ipsum dolor sit amet, consectetuer adipiscing elit. Aliquam
tempor tristique risus. Suspendisse potenti. Fusce id eros. In eu
enim. Praesent commodo leo. Nullam augue. Pellentesque tellus. Integer
pulvinar purus a dui convallis consectetuer. In adipiscing, orci vitae
lacinia semper, sapien elit posuere sem, ac euismod ipsum elit tempus
urna. Aliquam erat volutpat. Nullam suscipit augue sed
felis. Phasellus faucibus accumsan est. 

Aliquam felis justo, facilisis sit amet, bibendum ut, tempus ac,
dolor. Sed malesuada. Nunc non massa. In erat. Nulla
facilisi. Phasellus blandit, est in accumsan cursus, libero augue
elementum leo, vitae auctor mauris nisl ac tortor. Cras porttitor
ornare elit. Fusce at lorem. Sed lectus tortor, vestibulum id, varius
a, condimentum nec, lectus. Maecenas in nisi et magna pretium
aliquam. Pellentesque justo elit, feugiat nec, tincidunt a, dignissim
vel, ipsum. Sed nunc. Vestibulum ante ipsum primis in faucibus orci
luctus et ultrices posuere cubilia Curae; Aliquam tempus rhoncus
leo. Donec neque quam, cursus sit amet, ultricies varius, semper non,
pede. Donec porttitor. Sed aliquet feugiat elit.  

\vspace*{12mm}

Lorem ipsum dolor sit amet, consectetuer adipiscing elit. Phasellus
tellus pede, auctor ut, tincidunt a, consectetuer in, felis. Mauris
quis dolor et neque accumsan pellentesque. Donec dui magna,
scelerisque mattis, sagittis nec, porta quis, nulla. Vivamus quis
nisl. Etiam vitae nisl in diam vehicula viverra. Sed sollicitudin
scelerisque est. Nunc dapibus. Sed urna. Nulla gravida. Praesent
faucibus, risus ac lobortis dignissim, est tortor laoreet mauris,
dictum pellentesque nunc orci tincidunt tellus. Nullam pulvinar, leo
sed vestibulum euismod, ante ligula elementum pede, sit amet dapibus
lacus tortor ac nisl. Morbi libero. Integer sed dolor ac lectus
commodo iaculis. Donec ut odio.  
 

%%----------------------------------------
%% Final materials
%%----------------------------------------

%% Bibliography
%% Comment the next command if BibTeX file not used, 
%% Assumes that bibliography is in ``myrefs.bib''
\PrintBib{myrefs}

%% Comment next 2 commands if numbered appendices are not used
\appendix
\include{appendix1}

%% Index
%% Uncomment next command if index is required, 
%% don't forget to run ``makeindex pdis'' command
%\PrintIndex

\end{document}
