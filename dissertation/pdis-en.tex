%% FEUP THESIS STYLE for LaTeX2e
%% how to use feupteses (English version)
%%
%% FEUP, JCL & JCF, 31 July 2012
%%
%% PLEASE send improvements to jlopes at fe.up.pt and to jcf at fe.up.pt
%%

%%========================================
%% Commands: pdflatex tese
%%           bibtex tese
%%           makeindex tese (only if creating an index)
%%           pdflatex tese
%% Alternative:
%%          latexmk -pdf tese.tex
%%========================================

\documentclass[11pt,a4paper,twoside,openright]{report}

%% For iso-8859-1 (latin1), comment next line and uncomment the second line
\usepackage[utf8]{inputenc}
%\usepackage[latin1]{inputenc}

%% English version

%% MIEIC options
\usepackage[mieic]{feupteses}
%\usepackage[mieic,juri]{feupteses}
%\usepackage[mieic,final]{feupteses}
%\usepackage[mieic,final,onpaper]{feupteses}

%% Additional options for feupteses.sty: 
%% - onpaper: links are not shown (for paper versions)
%% - backrefs: include back references from bibliography to citation place

%% Uncomment the next lines if side by side graphics used
%\usepackage[lofdepth,lotdepth]{subfig}
%\usepackage{graphicx}
%\usepackage{float}

%% Include color package
\usepackage{color}
\definecolor{cloudwhite}{cmyk}{0,0,0,0.025}

%% Include source-code listings package
\usepackage{listings}
\lstset{ %
 language=C,                        % choose the language of the code
 basicstyle=\footnotesize\ttfamily,
 keywordstyle=\bfseries,
 numbers=left,                      % where to put the line-numbers
 numberstyle=\scriptsize\texttt,    % the size of the fonts that are used for the line-numbers
 stepnumber=1,                      % the step between two line-numbers. If it's 1 each line will be numbered
 numbersep=8pt,                     % how far the line-numbers are from the code
 frame=tb,
 float=htb,
 aboveskip=8mm,
 belowskip=4mm,
 backgroundcolor=\color{cloudwhite},
 showspaces=false,                  % show spaces adding particular underscores
 showstringspaces=false,            % underline spaces within strings
 showtabs=false,                    % show tabs within strings adding particular underscores
 tabsize=2,	                    % sets default tabsize to 2 spaces
 captionpos=b,                      % sets the caption-position to bottom
 breaklines=true,                   % sets automatic line breaking
 breakatwhitespace=false,           % sets if automatic breaks should only happen at whitespace
 escapeinside={\%*}{*)},            % if you want to add a comment within your code
 morekeywords={*,var,template,new}  % if you want to add more keywords to the set
}

%% Uncomment to create an index (at the end of the document)
%\makeindex

%% Path to the figures directory
%% TIP: use folder ``figures'' to keep all your figures
\graphicspath{{figures/}}

%%----------------------------------------
%% TIP: if you want to define more macros, use an external file to keep them
%some macro definitions

% format
\newcommand{\class}[1]{{\normalfont\slshape #1\/}}

% entities
\newcommand{\Feup}{Faculdade de Engenharia da Universidade do Porto}

\newcommand{\svg}{\class{SVG}}
\newcommand{\scada}{\class{SCADA}}
\newcommand{\scadadms}{\class{SCADA/DMS}}

%%----------------------------------------

%%========================================
%% Start of document
%%========================================
\begin{document}

%%----------------------------------------
%% Information about the work
%%----------------------------------------
\title{Title of the Dissertation}
\author{Name of the Author}

%% Uncomment next line for date of submission
%\thesisdate{July 31, 2008}

%%Uncomment next line for copyright text if used
%\copyrightnotice{Name of the Author, 2008}

\supervisor{Supervisor}{Name of the Supervisor}

%% Uncomment next line if necessary
%\supervisor{Second Supervisor}{Name of the Supervisor}

%% Uncomment committee stuff in the final version if used
%\committeetext{Approved in oral examination by the committee:}
%\committeemember{Chair}{Doctor Name of the President}
%\committeemember{External Examiner}{Doctor Name of the Examiner}
%\committeemember{Supervisor}{Doctor Name of the Supervisor}
%\signature

%% Specify cover logo (in folder ``figures'')
\logo{uporto-feup.pdf}

%% Uncomment next line for additional text  below the author's name (front page)
\additionalfronttext{Dissertation Planning}

%%----------------------------------------
%% Preliminary materials
%%----------------------------------------

% remove unnecssary \include{} commands
\begin{Prolog}
  \include{abstract_en} % the abstract
  \include{acknows_en}  % the acknowledgments
  \include{quote}       % initial quotation if desired
  \cleardoublepage
  \pdfbookmark[0]{Table of Contents}{contents}
  \tableofcontents
  \cleardoublepage
  \pdfbookmark[0]{List of Figures}{figures}
  \listoffigures
  \cleardoublepage
  \pdfbookmark[0]{List of Tables}{tables}
  \listoftables
  \include{abbrevs_en}  % the list of abbreviations used
\end{Prolog}

%%----------------------------------------
%% Body
%%----------------------------------------
\StartBody

%% TIP: use a separate file for each chapter
\include{chapter1} 
\include{chapter2}
\include{chapter3}
\include{chapter4}
\chapter{Conclusões e Trabalho Futuro} \label{chap:concl}

\section*{}

Deve ser apresentado um resumo do trabalho realizado e apreciada a
satisfação dos objetivos do trabalho, uma lista de contribuições
principais do trabalho e as direções para trabalho futuro.

A escrita deste capítulo deve ser orientada para a total compreensão
do trabalho, tendo em atenção que, depois de ler o Resumo e a
Introdução, a maioria dos leitores passará à leitura deste capítulo de
conclusões e recomendações para trabalho futuro.

\section{Satisfação dos Objetivos}

Lorem ipsum dolor sit amet, consectetuer adipiscing elit. Etiam non
felis sed odio rutrum ultrices. Donec tempor dolor. Vivamus justo
neque, tempus id, ullamcorper in, pharetra non, tellus. Praesent eu
orci eu dolor congue gravida. Sed eu est. Donec pulvinar, lectus et
eleifend volutpat, diam sapien sollicitudin arcu, a sagittis libero
neque et dolor. Nam ligula. Cras tincidunt lectus quis nunc. Cras
tincidunt congue turpis. Nulla pede velit, sagittis a, faucibus vitae,
porttitor nec, ante. Nulla ut arcu. Cras eu augue at ipsum feugiat
hendrerit. Proin sed justo eu sapien eleifend elementum. Pellentesque
habitant morbi tristique senectus et netus et malesuada fames ac
turpis egestas. Vivamus quam lacus, pharetra vel, aliquam vel,
volutpat sed, nisl. 

Nullam erat est, vehicula id, tempor non, scelerisque at,
tellus. Pellentesque tincidunt, ante vehicula bibendum adipiscing,
lorem augue tempor felis, in dictum massa justo sed metus. Suspendisse
placerat, mi eget molestie sodales, tortor ante interdum dui, ac
sagittis est pede et lacus. Duis sapien. Nam ornare turpis et
magna. Etiam adipiscing adipiscing ipsum. Fusce sodales nisl a
arcu. Cras massa leo, vehicula facilisis, commodo a, molestie
faucibus, metus. Suspendisse potenti. Duis sagittis. Donec porta. Sed
urna. Maecenas eros. Vivamus erat ligula, pharetra sit amet, bibendum
et, fermentum sed, dolor. Nullam eleifend condimentum nibh. Integer
leo nibh, consequat eget, mollis et, sagittis ac, felis. Duis viverra
pede in pede. Phasellus molestie placerat leo. Praesent at tellus a
augue congue molestie. Proin sed justo eu sapien eleifend
elementum. Pellentesque habitant morbi tristique senectus et netus et
malesuada fames ac turpis egestas. 

\section{Trabalho Futuro}

Lorem ipsum dolor sit amet, consectetuer adipiscing elit. Aliquam
tempor tristique risus. Suspendisse potenti. Fusce id eros. In eu
enim. Praesent commodo leo. Nullam augue. Pellentesque tellus. Integer
pulvinar purus a dui convallis consectetuer. In adipiscing, orci vitae
lacinia semper, sapien elit posuere sem, ac euismod ipsum elit tempus
urna. Aliquam erat volutpat. Nullam suscipit augue sed
felis. Phasellus faucibus accumsan est. 

Aliquam felis justo, facilisis sit amet, bibendum ut, tempus ac,
dolor. Sed malesuada. Nunc non massa. In erat. Nulla
facilisi. Phasellus blandit, est in accumsan cursus, libero augue
elementum leo, vitae auctor mauris nisl ac tortor. Cras porttitor
ornare elit. Fusce at lorem. Sed lectus tortor, vestibulum id, varius
a, condimentum nec, lectus. Maecenas in nisi et magna pretium
aliquam. Pellentesque justo elit, feugiat nec, tincidunt a, dignissim
vel, ipsum. Sed nunc. Vestibulum ante ipsum primis in faucibus orci
luctus et ultrices posuere cubilia Curae; Aliquam tempus rhoncus
leo. Donec neque quam, cursus sit amet, ultricies varius, semper non,
pede. Donec porttitor. Sed aliquet feugiat elit.  

\vspace*{12mm}

Lorem ipsum dolor sit amet, consectetuer adipiscing elit. Phasellus
tellus pede, auctor ut, tincidunt a, consectetuer in, felis. Mauris
quis dolor et neque accumsan pellentesque. Donec dui magna,
scelerisque mattis, sagittis nec, porta quis, nulla. Vivamus quis
nisl. Etiam vitae nisl in diam vehicula viverra. Sed sollicitudin
scelerisque est. Nunc dapibus. Sed urna. Nulla gravida. Praesent
faucibus, risus ac lobortis dignissim, est tortor laoreet mauris,
dictum pellentesque nunc orci tincidunt tellus. Nullam pulvinar, leo
sed vestibulum euismod, ante ligula elementum pede, sit amet dapibus
lacus tortor ac nisl. Morbi libero. Integer sed dolor ac lectus
commodo iaculis. Donec ut odio.  
 

%%----------------------------------------
%% Final materials
%%----------------------------------------

%% Bibliography
%% Comment the next command if BibTeX file not used
%% bibliography is in ``myrefs.bib''
\PrintBib{myrefs}

%% comment next 2 commands if numbered appendices are not used
\appendix
\include{appendix1}

%% Index
%% Uncomment next command if index is required
%% don't forget to run ``makeindex pdis-en'' command
%\PrintIndex

\end{document}
