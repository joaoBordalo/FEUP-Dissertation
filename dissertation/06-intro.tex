\chapter{Introduction} \label{chap:intro}

\section*{}
\section{Context} \label{sec:context}

Previously, computer systems were built to maximize their processing power in compactness and individually because programs were developed with a sequential approach. With the advance in microchips' technology, computers increased their processing capacity per volume, however some issues arose, such as high energy cost, high temperature and low equipment durability. To solve these issues some measures needed to take place in order to make computers systems more reliable, durable, efficient, and powerful.

Recently, the computing industry has moved away from exponential scaling of clock frequency toward chip multiprocessors in order to better manage trade-offs among performance, energy efficiency, and reliability~\cite{Datta2008}

Combining different computer processing components, such as CPU, GPU Xeon Phi and FPGA, in a single computer system removed some heavy burden in the main processing core, making the computer system with better performance and reliable. However some concerns arose: how to properly use these components without jeopardizing the computer system and application performance. Some processing components can handle specif jobs better then others and combined the computer can achieve a whole new performance level; for instance, the use of a GPU together with a CPU to accelerate deep learning algorithm, analytics, and engineering applications~\cite{NvidiaGPU}, however this kind of utility is not yet well optimized and its utility is only recently emerging.


- mencionar multiplicação de matrizes,
-dois algoritmos ligeiramente diferentes para a multiplicação





%\section{Projeto} \label{sec:proj}

% Na continuação da secção anterior, e apenas no caso de ser um Projeto
% e não uma Dissertação, esta secção apresenta resumidamente o projeto.

% Nulla nec eros et pede vehicula aliquam. Aenean sodales pede vel
% ante. Fusce sollicitudin sodales lacus. Maecenas justo mauris,
% adipiscing vitae, ornare quis, convallis nec, eros. Etiam laoreet
% venenatis ipsum. In tellus odio, eleifend ac, ultrices vel, lobortis
% sed, nibh. Fusce nunc augue, dictum non, pulvinar sed, consectetuer
% eu, ipsum. Vivamus nec pede. Pellentesque pulvinar fringilla dolor. In
% sit amet pede. Proin orci justo, semper vel, vulputate quis, convallis
% ac, nulla. Nulla at justo. Mauris feugiat dolor. Etiam posuere
% fermentum eros. Morbi nisl ipsum, tempus id, ornare quis, mattis id,
% dolor. Aenean molestie metus suscipit dolor. Aliquam id lectus sed
% nisl lobortis rhoncus. Curabitur vitae diam sed sem aliquet
% tempus. Sed scelerisque nisi nec sem. 

\section{Motivation and Goal} \label{sec:goals}

My motivation for this thesis is to advance a little further on the field of the automatic code parallelization  and replace the manual parallelization labor because it requires a lot of time and effort to achieve significant performance.


\section{Structure of the Report} \label{sec:struct}

% Para além da introdução, esta dissertação contém mais x capítulos.
% No capítulo~\ref{chap:sota}, é descrito o estado da arte e são
% apresentados trabalhos relacionados. 
% %\todoline{Complete the document structure.}
% No capítulo~\ref{chap:chap3}, ipsum dolor sit amet, consectetuer
% adipiscing elit.
% No capítulo~\ref{chap:chap4} praesent sit amet sem. 
% No capítulo~\ref{chap:concl}  posuere, ante non tristique
% consectetuer, dui elit scelerisque augue, eu vehicula nibh nisi ac
% est. 

This report is divided in three more chapters. The next one is called \textit{Achieving the Highest Processing Power}, and it is related to the state of the art of my thesis' scope. In this chapter there are three sections. The first section is related to to the context of the state of the art in the filed. The other two sections are two different but complementary approaches which help and describe the state of the art. 

The third chapter addresses the problem involved in my thesis and how I propose to solve it, including the approach, the methodology and solution's validation. The last chapter  includes final consideration related to the work developed so far in \textit{Preparação da Dissertação} course, expected results with my proposed solution and work plan to develop the solution. In the end of this report there are the references used to develop this report.
